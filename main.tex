\documentclass[notheorems, hyperref]{beamer}

%% KEY LINES IN THIS TEX FILE %% (enter line number+gg to go)

%
% LOCAL FONT DEFINITIONS -- need to come first
%
%\usepackage{mathpazo}
%\usepackage{libertine}
%\usepackage[libertine]{newtxmath}
%\usefonttheme[onlymath]{serif}

%
% STANDARD PREAMBLE
%
\input{preamble}
\allowdisplaybreaks

%
% ABOUT FONT DEFINTIONS IN THE PREAMBLE
%
% Mathscr for sheaves use \sA, where A can be any letter. Exceptions and additions:
% % \E (vector bundles)
% % \F (coherent sheaves)
% % \G (coherent sheaves)
% % \hom (sheaf hom)
% % \I (ideal sheaves)
% % \L (line bundles)
% % \M (line bundles)
% % \O (structure sheaf)
% % \w (canonical sheaf)
%
% Mathcal use \calA. Exceptions and additions:
% % \U (open cover)
% % \X (families of varieties)
% % \Y (families of varieties)
%
% Mathbb use \bbA. Exceptions and additions:
% % \A (affine space)
% % \C (complex numbers)
% % \Gm (puctured affine line)
% % \k (field)
% % \N (natural numbers)
% % \P (projective space)
% % \Q (rational numbers)
% % \R (real numbers)
% % \V (geometric vector bundle)
% % \Z (integers)
%
% Boldfont for categories use \bfA. Additions:
% % \Cat (categories)
% % \Coh (coherent sheaves)
% % \D (derived category)
% % \Db (bounded derived category)
% % \K (homotopy category)
% % \Mod (modules)
% % \PSh (presheaves)
% % \QCoh (quasi-coherent sheaves)
% % \Set (sets)
% % \Sh (sheaves)
% % \Top (topological spaces)
% % \Vec (vector bundles)
%
% Mathfrak for ideals
% % From \a to \e
% % \m and \n for maximal ideals

%
% THEOREM ENVIRONMENTS
%
% Theorems, propositions, etc (dark green)
\theoremstyle{darkgreentheorem}
\newtheorem{thm}{Theorem}
\newtheorem{reform}{Reformulation}
\newtheorem{lm}[thm]{Lemma}
\newtheorem{prop}[thm]{Proposition}
\newtheorem{cor}[thm]{Corollary}
\newtheorem{conj}[thm]{Conjecture}
% Definitions (dark blue)
\theoremstyle{darkbluedefinition}
\newtheorem{defn}[thm]{Definition}
% Examples (dark red)
\theoremstyle{darkredexample}
\newtheorem{exa}[thm]{Example}
% Remarks (black)
\theoremstyle{remark}
\newtheorem{rem}[thm]{Remark}
\newtheorem{nota}[thm]{Notation}
\newtheorem{fact}[thm]{Fact}
\newtheorem{q}[thm]{Question}
\newtheorem{pbl}[thm]{Problem}

%
% THEOREM CROSS-REFERENCING
%
\crefname{thm}{theorem}{theorems}
\Crefname{thm}{Theorem}{Theorems}
\crefname{lm}{lemma}{lemmas}
\Crefname{lm}{Lemma}{Lemmas}
\crefname{prop}{proposition}{propositions}
\Crefname{prop}{Proposition}{Propositions}
\crefname{cor}{corollary}{corollaries}
\Crefname{cor}{Corollary}{Corollaries}
\crefname{conj}{conjecture}{conjectures}
\Crefname{conj}{Conjecture}{Conjectures}
\crefname{defn}{definition}{definitions}
\Crefname{defn}{Definition}{Definitions}
\crefname{exa}{example}{examples}
\Crefname{exa}{Example}{Examples}
\crefname{rem}{remark}{remarks}
\Crefname{rem}{Remark}{Remarks}
\crefname{nota}{notation}{notations}
\Crefname{nota}{Notation}{Notations}
\crefname{fact}{fact}{facts}
\Crefname{fact}{Fact}{Facts}
\crefname{q}{question}{questions}
\Crefname{q}{Question}{Questions}
\crefname{pbl}{problem}{problems}
\Crefname{pbl}{Problem}{Problems}

%
% MATH OPERATORS
%
\DeclareMathOperator{\Hom}{Hom}
\DeclareMathOperator{\Ker}{Ker}
\DeclareMathOperator{\Pic}{Pic}

%
% OTHER COMMANDS
%
\newcommand{\ot}{\otimes}
\newcommand{\op}{\oplus}
\renewcommand{\L}{\mathcal{L}}
\renewcommand{\M}{\mathcal{M}}
\newcommand{\dual}{^{\vee}}
\newcommand{\av}{\mathbf{AV}}
\newcommand{\avi}{\mathbf{AV}^{0}}

%
% TITLE PAGE INFORMATION
%
\title[Endomorphisms of abelian varieties]{Endomorphisms of abelian varieties}
\author{Remarks on Section I.10 of Milne's \textit{Abelian Varieties}}
\institute{University of Freiburg}
\date{23th June 2020}
 
%
% LINKS AND PDF OPTIONS
%
\makeatletter
\hypersetup{
  %pdfauthor={\authors},
  pdftitle={\@title},
  %pdfsubject={\@subjclass},
  %pdfkeywords={\@keywords},
  %pdfstartview={Fit},
  %pdfpagelayout={TwoColumnRight},
  %pdfpagemode={UseOutlines},
  bookmarks,
  colorlinks,
  linkcolor=linkblue,
  citecolor=linkred,
  urlcolor=linkred}
\makeatother
\usecolortheme{rose}
 
\begin{document}
 
\frame{\titlepage}

\begin{frame}
    \frametitle{Recall isogenies}
    \begin{tcolorbox}[colback=blue!5!white,colframe=blue!75!black,title=Definition]
	\vspace{-6mm}
	\begin{align*}
	    \alpha\in \Hom(A,B) \text{ \textbf{isogeny} } & \text{\textcolor{blue}{$\Leftrightarrow$} surjective with finite kernel;} \\ 
	    & \text{\textcolor{blue}{$\Leftrightarrow$} surjective and $\dim{A}=\dim{B}$;} \\
	    & \text{\textcolor{blue}{$\Leftrightarrow$} finite kernel and $\dim{A}=\dim{B}$;} \\
	    & \text{\textcolor{blue}{$\Leftrightarrow$} finite and surjective (and flat).}
	\end{align*}
    \end{tcolorbox}
    \pause
    \begin{tcolorbox}[colback=red!5!white,colframe=red!75!black,title=Examples]
	\begin{itemize}
	    \item $\L$ ample $\Rightarrow$ $\lambda_{\L}\colon A\to A\dual$ isogeny \cite[Prop.~8.1]{mil08}.
	    \item $n>0\Rightarrow n_{A}\colon a\mapsto na$ isogeny \cite[Thm.~7.2]{mil08}.
	\end{itemize}
    \end{tcolorbox}
    \pause
    Last example $\Rightarrow n$-torsion subgroup $A_{n}:=\ker(a\mapsto na)$ is \textbf{finite}.
\end{frame}

\begin{frame}
    \frametitle{Abelian varieties up to isogeny}
    \begin{itemize}
	\item $\av(k)$: additive cat.~of abelian varieties and regular homomorphisms over a field $k$.
	\item $\avi(k)$: $\Q$-linear cat.~with $\Hom^{0}(A,B):=\Hom(A,B)\ot_{\Z}\Q$.
	\item $\alpha\in \Hom(A,B)$ isogeny $\Rightarrow B\cong A/\Ker(\alpha)$ and $\Ker(\alpha)\subseteq A_{n}$ for some $n\in \N$.
	    Hence $\alpha$ becomes an isomorphism in $\avi(k)$:
	    \begin{center}
		\begin{tikzcd}[ampersand replacement=\&]
		    A\arrow[swap]{dr}{\alpha}\arrow{rr}{n_{A}} \& \& A \\
		    \& A/\Ker(\alpha)\arrow[dashed,swap]{ur}{\exists} \&
		\end{tikzcd}
	    \end{center}
	\item $\alpha\in \Hom(A,B)$ isomorphism in $\Hom^{0}(A,B)\Rightarrow $ we can find such a factorization.
	\item In fact $\av(k)\to \avi(k)$ localizes $\av(k)$ at all isogenies.
    \end{itemize}
\end{frame}

\begin{frame}
    \frametitle{The category $\avi(k)$}
    \begin{itemize}
	\item It is \textbf{abelian} [\href{https://encyclopediaofmath.org/wiki/Isogeny}{encyclopediaofmath.org/wiki/Isogeny}].
	\item Every object is \textbf{semisimple}: every $A$ is isogenous to a finite direct sum of indecomposable $A_{i}$'s \cite[Prop.~I.10.1]{mil08}.
	    \begin{itemize}
		\item Thus every short exact sequence in $\avi(k)$ splits [Jeremy Rickard's comment on \href{https://mathoverflow.net/a/327944/99436}{mathoverflow.net/a/327944/99436}].
		\item Thus all additive functors are already exact and its derived category, which is equivalent to the category of cochain complexes with trivial differentials, is abelian \cite[III.2.4]{gm03}.
	    \end{itemize}
	\item All homs are finite dimensional over $\Q$ \cite[Thm.~I.10.15]{mil08}.
    \end{itemize}


    \begin{tcolorbox}[colback=yellow!5!white,colframe=orange!75!black]
    Semisimple objects $\Rightarrow $ to understand $\avi(k)$, it suffices to study endomorphism algebras of simple objects \cite[p.~43]{mil08}.	
    \end{tcolorbox}
\end{frame}

\begin{frame}
    \frametitle{References}
    \bibliographystyle{alpha}
    \bibliography{main.bib}
\end{frame}

\end{document}
